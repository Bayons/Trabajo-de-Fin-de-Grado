\documentclass{subfiles}


\begin{document}

    \begin{abstractpage}
        \addcontentsline{toc}{chapter}{\protect\numberline{}Resumen}
        \begin{abstract-lang}{spanish}
            Mediante este proyecto, se presenta la oportunidad al usuario de atender a una pequeña charla impartida por un asistente virtual dotado de forma humana que, aplicando la Realidad Aumentada, se ubicará en el mundo real a través de las imágenes captadas por la cámara del dispositivo móvil. De esta manera, y combinando varias tecnologías de software libre, siendo la interfaz \webxr la más destacable, se logrará generar una experiencia inmersiva para el usuario que logre captar su atención y le ofrezca una presentación de la web y de la empresa para la que se ha generado distinta a lo que puede encontrar en otros lugares, siendo necesario únicamente un dispositivo \android y el navegador \googlechrome para los dispositivos de dicho sistema operativo.
        \end{abstract-lang}
    \end{abstractpage}

    \begin{abstractpage}
        \begin{abstract-lang}{english}
            Through this project, the user will have the opportunity to attend a short talk given by a virtual assistant with a human form that, applying Augmented Reality, will be located in the real world through the images captured by the camera of the mobile device. In this way, and combining several free software technologies, being the \webxr interface the most remarkable one, it will be possible to generate an immersive experience for the user that will capture his attention and will offer him an introduction to the web and the company for which it has been generated different from what he can find in other places, being necessary only an \android device and the \googlechrome browser for the devices of that operating system.
        \end{abstract-lang}
    \end{abstractpage}

\end{document}