\documentclass{subfiles}

\begin{document}

    \chapter{Consideraciones y futuros pasos}
    \label{chap:6}

    \section{Pasos fallidos}
    \label{sec:6.1}
    
    Es importante remarcar que este trabajo forma parte de una colaboración con la empresa \thirdera, originalmente \silverstorm, y que algunos recursos fueron ofrecidos por esta misma para favorecer el curso del desarrollo y su correcta finalización, así como asegurar el producto mínimo viable.

    \paragraph{}
    Dicho esto, también hay que tener en cuenta que la empresa no está especializada en desarrollos de este tipo, y por lo tanto todos éramos nuevos usando las tecnologías aplicadas en el proyecto. Por eso, pese a haber gestionado riesgos y organizado el trabajo, inevitablemente surgieron problemas que no pudimos abordar por diferentes motivos. Los más importantes fueron los siguientes:

    \begin{itemize}
        \item Aunque se pensó desde un primer momento en la compatibilidad entre distintos dispositivos móviles, lo cierto es que fue un punto que no se consiguió. Originalmente se pensó que el uso de \textit{WebXRPolyfill} permitiría utilizar la \ra en diferentes navegadores debido a que hacen mención a compatibilidad en navegadores en su documentación \cite{web:webxrpolyfill} (a pesar de que no hacen mucho énfasis en ello). Pero, finalmente, resultó que esta actuaba como una <<extensión>> para aplicar retrocompatibilidad en caso de que el usuario utilizara un navegador \googlechrome para móviles con una versión antigua. La compatibilidad con otras plataformas fue finalmente descartada debido a que no encontramos una solución alternativa.
        \item A la hora de generar las animaciones corporales de los modelos, se planteó utilizar la herramienta web \textit{Kinetix}: una aplicación que capta los movimientos de una persona, los traduce a movimientos en el esqueleto de un modelo 3D y devuelve el modelo con la animación completa \cite{web:kinetix}. Por desgracia, esta aplicación no reconoce correctamente los esqueletos de modelos 3D que no hayan sido generados por la misma aplicación, por lo que el resultado finalmente era totalmente incorrecto y no se podía considerar utilizable. Debido al consumo de tiempo que implicaba buscar otra herramienta y aprender a utilizarla, se cambió este paso por el desarrollo <<a mano>> de las animaciones corporales del modelo a través de \blender.
        \item Cuando se generaron originalmente los modelos, se crearon directamente con la pose de reposo que tienen actualmente cuando no se están moviendo. Sin embargo, en numerosas webs de diseño de modelos 3D se recomienda encarecidamente generarlos en pose A o en pose T, es decir, con todas las extremidades estiradas y con los brazos estirados hacia las piernas (pose A) o estirados en pose de cruz (pose T). Esto se hace para que los movimientos de los huesos sean mucho más precisos a la hora de mover cada miembro. Además, en este caso, hubiera favorecido que las mallas de los modelos se deformaran menos, ya que \makehuman, al generar un modelo con una pose concreta, también reestructura las mallas para acomodarlas a la postura elegida. El elegir una postura distinta a las poses A y T como postura de reposo ha generado algunas deformidades en brazos y dedos a la hora de moverlos y la imposibilidad de mover de manera normal los labios de los modelos.
        \item Una de las características que contiene \webxr y que no aparecía en ninguno de los tutoriales de primeros pasos de Google era las capas de \textit{renderState}. Estas capas están preparadas para que cada una contenga una imagen, de manera que se superpongan las distintas capas y se genere el efecto de \ra. Esto favorecería un desarrollo más limpio y organizado del código, pero no se utilizó debido al desconocimiento del mismo. En su lugar, se utilizó una sola capa en la que se implantaban todas las imágenes de manera ordenada.
    \end{itemize}

    \section{Futuros pasos}
    \label{sec:6.2}
    Durante el desarrollo y después de finalizar el mismo, se encontraron muchos puntos que necesitan una corrección o que pueden mejorarse. Algunos de estos requieren de mejoras en las propias interfaces utilizadas, por lo que su mejora pasaría por esperar a actualizaciones o sustituir completamente la interfaz, pero otras son puntos mejorables que se han ido encontrando durante el proyecto y posponiendo hasta después de obtener un producto mínimo viable. Algunos de los puntos encontrados son los siguientes:
    \begin{itemize}
        \item En el código original, se utiliza la opción \textit{preserveDrawingBuffer} como una de las opciones a añadir al renderizador. Esto se utilizó debido a que viene así explicado en varios tutoriales, pero en un issue del repositorio de GitHub de \webxr indican que esta opción es totalmente ignorada por la herramienta, por lo que no tiene ningún sentido mantener esa opción en el objeto.\cite{web:webxr_github_preserveDrawingBuffer}. Este hilo es el único lugar donde se indica esta información, hecho por el cuál costó tanto encontrar este dato.
        \item La detección de superficies resulta muy imprecisa y depende, como se comentó en la sección \ref{sec:5.2}, de la iluminación y del tipo de superficie. Según parece, la detección de superficies se puede configurar, por lo que es posible que se pueda mejorar manualmente, aunque esto requiere de mucha investigación.
        \item Como se comentó en la sección anterior, la aplicación no es multiplataformas, sino que funciona solo en determinados navegadores específicos determinados por la propia librería \webxr. En futuros pasos, sería ideal ampliar esta compatibilidad, ya sea configurando el sistema actual de la manera adecuada o cambiando de librería de \ra.
        \item Los pasos para añadir nuevos modelos, animaciones y voces se encuentran muy medidos y son reproducibles. Sin embargo, sería ideal simplificarlos aún más para poder minimizar el esfuerzo de generar nuevo contenido para la aplicación.
        \item A pesar de que el código del sistema está repartido para repartir responsabilidades, hubiera sido posible diferenciarlo más de manera que los diferentes trabajos, especialmente en el bucle de procesado, estuviesen mucho más diferenciados y fuese más legible. Esto requeriría de una refactorización intensa, pero que puede afectar positivamente en el mantenimiento de la propia aplicación.
        \item Algunos sistemas basados en \webxr son capaces de detectar objetos que se encuentran entre el modelo y la cámara. Al reconocer esto, dichos sistemas son capaces de <<ocultar>> el modelo detrás del objeto, cosa que el sistema actual no es capaz de hacer. Sería ideal encontrar la manera de que los modelos se puedan <<esconder>> detrás de objetos reales.
        \item La orientación inicial del modelo, a día de hoy, depende del momento de carga del mismo: este mirará siempre hacia el punto en que se encontraba el usuario en el momento exacto de carga. El sistema sería mucho más vistoso si, cada vez que se pulsara sobre la pantalla, el modelo se girase hacia la cámara del usuario.
        \item Como último punto detectado, la accesibilidad de la aplicación mejoraría si esta mostrase al usuario las instrucciones para su uso en el momento correcto. Es decir: que la aplicación mostrase mensajes como <<mueva la cámara para detectar superficies>>, <<toque la pantalla para comenzar la animación>> o <<toque de nuevo la pantalla para reiniciar la aplicación>> en los momentos indicados para que, en todo momento, el usuario sepa cómo utilizar este sistema.
    \end{itemize}

\end{document}