\documentclass{subfiles}

\begin{document}

  \chapter{Sistema básico}
  \label{chap:2}

        \section{Sesión WebXR}
        \label{sec:2.1}
        
        
        \section{Procesado iterativo}
        \label{sec:2.x}

        Para poder asignar a cada fotograma, así como hacer los propios cálculos en cada momento de la ubicación y movimientos de las figuras que vamos a insertar llegado el momento, es necesario crear un bucle que dure tanto tiempo como el usuario vaya a permanecer en la aplicación. En cada una de las iteraciones, la aplicación solicitará al hardware del dispositivo móvil la imagen captada a través de su cámara, de disponer de una.

        \paragraph{}
        Para evitar que el sistema se ralentice, las solicitudes de fotogramas se realizan de una forma un tanto particular. En lugar de utilizar un bucle estándar como pueda ser uno tipo <<for>> o <<while>>

\end{document}
