\documentclass{subfiles}

\begin{document}

    \chapter{Introducción}
    \label{chap:1}

        \section{Motivación}
        \label{sec:1.1}

        {El \TFG supone el último paso para obtener la titulación cursada, y se muestra de manera muy intimidante al alumno. Por mucho que otros compañeros de años anteriores intenten suavizar su dificultad o la necesidad de hacer algo grande, uno no deja de verlo como la culminación de varios años de aprendizaje en un estudio superior. Por eso, fue difícil elegir algo que realizar.}
      
        \paragraph{}
        {Durante varios años, estuve dispuesto a llevar mi propio proyecto con varias ideas que fui reservando para más adelante, e incluso fueron varias las revisiones que hice a las ideas preparadas para distintos trabajos por distintos profesores. Sin embargo, a medida que se iba acercando la fecha de comenzar este proyecto, iba creciendo la necesidad de comenzar a trabajar y tener mis propios ingresos. Así, y teniendo el \TFG en mente, comencé a buscar empleo planteando a las empresas que me concedían entrevista dos condiciones: media jornada para poder terminar mis estudios y que me ofrecieran un proyecto válido para un \tfg para poder terminar mi carrera.}
      
        \paragraph{}
        {Fue así como entré en \silverstorm, ahora \thirdera después de su adquisición por parte de esta última. Ellos me concedieron la flexibilidad necesaria para compaginar empleo y estudios, así como facilidades a la hora de realizar los exámenes de la Universidad. Además, al entrar en el equipo de innovación, sabía que me podrían ofrecer un proyecto que se distinguiese de los que ofrecían al resto de estudiantes que trabajaban en la empresa.}
        
        \paragraph{}
        {Tras un tiempo, y después de comentarles que quería comenzar a desarrollar mi \TFG, me ofrecieron el proyecto que se explicará en esta memoria. Por aquel entonces, la página web de \silverstorm desarrollada en \wordpress contaba con un estilo sencillo y familiar para que los usuarios que entrasen conociesen la empresa y pudieran obtener la información que desearan de ella: a qué nos dedicamos, cómo y con qué trabajamos, qué clientes tenemos o hemos tenido, de qué ofertas de trabajo disponemos, etc. No obstante, esta no lograba destacar de otras webs corporativas: sus animaciones eran aquellas que ofrecía \wordpress, los dibujos utilizados eran de una colección de imágenes de stock retocadas y su sencillez implicaba una cierta monotonía.}
        
        \paragraph{}
        {La idea del proyecto era añadir un pequeño detalle a esta página que hiciera que esta destacara un poco más. La propuesta consistía en generar un pequeño avatar que pudiese contar, en cada una de las secciones, lo que el usuario podía encontrar. Este avatar aparecería en el plano real mediante \ra, una tecnología en pleno auge \cite{Xiong2021} que reforzaría la idea de que la empresa se mantiene en contacto con las últimas tendencias tecnológicas. Además, para facilitar el uso de la cámara para ofrecer imágenes del mundo real, era imprescindible que el usuario accediese a esta experiencia a través del móvil.}

        \section{Objetivos}
        \label{sec:1.2}

        {Debido a que el trabajo con estas tecnologías era nuevo para nuestro equipo y carecíamos de experiencia al respecto, el primer punto a abordar fue la investigación sobre estas tecnologías. En primer lugar, se tomaron como referencia algunas aplicaciones publicitarias que utilizaban unas versiones algo más rudimentarias de \ra donde, escaneando un código QR, aparecía en pantalla una persona comentando promociones sobre la imagen grabada por la cámara del móvil. Nuestra intención era investigar qué tecnología se usó en ese caso o en otros casos donde el resultado fuese mejor que el de partida.}
        
        \paragraph{}
        {Por supuesto, y dado que el proyecto mayoritariamente trabajaría sobre la tecnología utilizada, el siguiente objetivo sería aprender a utilizarla y \textbf{generar una estructura eficaz} y escalable que permitiese futuras mejoras.}

        \paragraph{}
        {Para complementar la experiencia inmersiva de la aplicación, se añadiría \textbf{sonido espacial} al mismo sistema. Esto permitiría que un usuario con auriculares pudiera reconocer, sin mirar a la pantalla, la dirección de la que procedía el sonido y la distancia que le separaba del origen de este.}
        
        \paragraph{}
        {También tendríamos que \textbf{desarrollar un modelo en 3D} que funcionase como nuestro avatar. Para ello, habría que buscar herramientas de creación, diseño y modelado en 3D, además de aprender a utilizarlos para obtener un resultado satisfactorio.}
        
        \paragraph{}
        {Para que diese la impresión de que el modelo fuese el que dijera todo lo mencionado en la grabación, sería necesario \textbf{sincronizar la voz con los gestos faciales} y los labios del modelo. Para esta <<sincronización labial>>, sería más eficiente buscar una aplicación o herramienta que fuese fácil de configurar e hiciese gran parte del trabajo. Por lo tanto, este objetivo se podría dividir en buscar una herramienta de sincronización labial, aprender a utilizarla y desplegar su funcionalidad sobre el proyecto.}
        
        \paragraph{}
        {Sería necesario también \textbf{generar una animación} para el resto del cuerpo del modelo 3D, de manera que este no resultara demasiado estático al hablar y diese una impresión algo más realista.}
        
        \paragraph{}
        {El sistema web tiene que ser accesible a través de la web, por lo que sería necesario también alojarlo en un servidor que almacenase toda la funcionalidad, así como los protocolos necesarios para que la conexión fuese fiable y segura.}

        \paragraph{}
        {A mayores, otro punto a analizar en el proyecto era la accesibilidad del usuario a esta aplicación. Debido a la finalidad de la web, con obvias intenciones comerciales y laborales, el equipo encargado de desarrollar originalmente esta consideró que la mayor parte de los usuarios que visitaban la página accedían desde ordenadores, siendo una menor proporción los usuarios que accedían desde móviles o tablets. Por esta razón, se consideró que la mejor forma de hacer que un usuario pasara de un ordenador a un móvil era introduciendo un código QR que enlazase directamente a este sistema. En caso de que el usuario navegase desde el móvil, el mismo código funcionaría a la vez como un botón con la capacidad de redirigir.}


        \section{Metodología}
        \label{sec:1.3}
        
        {Debido a la capacidad de adaptación ante nuevos requisitos y a la posibilidad de generar nuevas entregas en un tiempo relativamente corto utilizando metodologías ágiles, se ha optado por esta preferiblemente sobre otras \cite{agilewebsite}. De esta manera, las partes interesadas serían capaces de ver en periodos cortos de tiempo avances en el proyecto que supongan un cambio importante. Es importante mencionar que, debido a que el equipo de desarrollo contaba solamente con una persona, no se ha aplicado el marco de trabajo \textit{Scrum}, puesto que este está planteado para equipos más grandes.}
        
        \paragraph{}
        {Para centrar el trabajo en el menor número de tareas posibles al mismo tiempo, y favorecer la visibilidad del estado de las tareas y del proyecto, se ha decidido utilizar una adaptación de la metodología \Kanban\cite{web:kanban}, una subcategoría de gestión ágil de proyectos centrada en el uso de los tableros homónimos. La idea de usar esta metodología surgió después de atender a una charla de Pablo Santos en la Escuela de Ingeniería Informática en la Universidad de Valladolid \cite{web:pablosantos}, donde nos explicó los diferentes proyectos que dirigió, así como las metodologías que aplicó para liderarlos. En el momento de la charla, explicó como la metodología \Kanban le permitía dar más visibilidad a las tareas en proceso, así como ver el flujo de trabajo, entre otros datos.}

        \begin{figure}
        \centering
        \includegraphics[width=0.8\textwidth]{kanbanboard}
        \caption{Ejemplo de tablero \Kanban. En este, las tarjetas pasan por varias columnas de trabajo en curso antes de finalizar.}
        \label{fig:kanbanboard}
        \end{figure}

        \paragraph{}
        {Según David Anderson, el ingeniero de Microsoft que aplicó esta metodología por primera vez a proyectos informáticos, los tableros \Kanban contienen cinco elementos característicos: señales visuales, columnas, límites de trabajo en curso, punto de compromiso y punto de entrega.}

        \begin{itemize}
            \item {\textbf{Señales visuales}: todo el trabajo, junto con su estado actual, debe ser visible de un solo vistazo. Esta metodología utiliza las tarjetas como representación de tareas únicas que, en combinación con las columnas, nos revelarán el flujo de trabajo y el estado actual de este último, todo esto sin abandonar la filosofía de <<información visual>> de \Kanban.}
            \item {\textbf{Columnas}: representan el estado de la tarea en el flujo de trabajo. Los tableros más sencillos suelen utilizar 3 columnas (\textit{pendiente}, \textit{en curso}, \textit{finalizado}), aunque cada equipo puede organizar su flujo de trabajo a placer. En su charla, Pablo Santos llegó a indicar que cada empleado tenía su propia columna de <<trabajo en curso>> para visibilizar de manera más eficaz con qué tarea se encontraba cada empleado.}
            \item {\textbf{Límites de trabajo en curso}: las columnas que indican el trabajo en curso deben tener un límite establecido de tarjetas en ella. La metodología establece que, cuando una de estas columnas alcanza su límite de tarjetas, el equipo debe aplicar todo su esfuerzo en dichas tarjetas para lograr que avancen. Esta es la forma visual que tiene \Kanban de mostrar al equipo que se está intentando asumir demasiado trabajo y de avisar de posibles cuellos de botella.}
            \item {\textbf{Punto de compromiso}: \Kanban trabaja con una pila de tareas por desarrollar, comúnmente llamada \textit{backlog}, que va creciendo según van apareciendo nuevas necesidades o requisitos. Cada vez que una de esas tareas sale de la columna de tareas pendientes, el miembro del equipo se <<compromete>> a terminar esa tarea en el menor tiempo posible. El momento de inicio de cada tarea se denomina por esto <<punto de compromiso>>}
            \item {\textbf{Punto de entrega}: es el momento en que una tarjeta se convierte en un trabajo completado y, por tanto, está ya en manos del cliente o del producto final. El tiempo entre el punto de compromiso y el punto de entrega se denomina <<plazo>>, y la intención de la forma de trabajar con \Kanban es que ese plazo sea el menor posible.}
        \end{itemize}
        
        \paragraph{}
        {Al ser un proyecto de innovación dentro de la empresa, los requisitos iban surgiendo según íbamos encontrando nueva información al respecto. De esta manera, durante reuniones semanales se comentaba el avance del proyecto, así como las nuevas tareas a incorporar junto con su prioridad, procurando siempre no solapar unas tareas con otras.}


        \section{Conceptos de Realidad Aumentada}
        \label{sec:1.4}

        Antes de comenzar a explicar en qué consiste este proyecto, es importante fijar algunos conceptos de \ra tal y como se van a utilizar en este tema, de manera que no queden ambigüedades a partir de este punto.

        \subsection{Realidad}
        \label{sec:1.4.1}
        El concepto más importante y a la vez básico a dejar claro es el concepto de \realidad. Para nuestro sistema, se considera \realidad al conjunto de objetos, información y estímulos, principalmente visuales, que componen nuestro entorno y que percibimos a través de nuestros sentidos. Así, si nos encontrásemos en nuestra casa, podríamos percibir varios elementos a través de la \realidad: una televisión, un sofá, una temperatura cálida, una iluminación baja, el ruido de un ventilador...

        \subsection[Realidad Aumentada, Realidad Virtual y Realidad Mixta]{Realidad Aumentada, Realidad Virtual y Realidad \\Mixta}
        \label{sec:1.4.2}

        La \ra y la \rv son dos conceptos que a menudo se confunden por ser relativamente nuevos y por sus puntos en común. Sin embargo, los resultados pueden llegar a ser muy distintos o incluso opuestos en algunos casos. En ambos existe información generada a través de sistemas informáticos, pero las diferencias son ampliamente notables.

        \begin{figure}[H]
        \centering
        \includegraphics[width=0.8\textwidth]{reality_continuum}
        \caption{Espectro entre un entorno real y un entorno virtual.}
        \label{fig:reality_continuum}
        \end{figure}

        \paragraph{}
        La diferencia más destacable es el concepto de \realidad anteriormente mencionado, así como su uso: en la \rv, todos los elementos son generados digitalmente, de manera que la \realidad percibida es totalmente distinta a nuestro entorno <<real>>. Para esto, comúnmente se hace uso de las gafas de \rv, que aíslan la visión del usuario para que solo pueda captar la información generada y transmitida por las gafas.
    
        \paragraph{}
        Sin embargo, en el caso de la \ra, la intención es ampliar en tiempo real los datos captados de la \realidad, por lo que la base en estas tecnologías siempre serán imágenes e información de nuestro entorno. A esto se le debe sumar todo aquello que añada el sistema para <<ampliar>> la \realidad: figuras, texto, imágenes, etc. Estas últimas se encontrarán superpuestas sobre lo captado del entorno de manera fidedigna y que ofrezca más valor a lo naturalmente captado.

        \begin{figure}
        \centering
        \includegraphics[width=0.8\textwidth]{vr_ar_devices}
        \caption{A la izquierda, las Meta Quest 2, las gafas de \rv de Meta. A la derecha, las Microsoft Hololens, las gafas de \ra de Microsoft.}
        \label{fig:vr_ar_devices}
        \end{figure}
    
        \paragraph{}
        Otra gran diferencia se puede encontrar en los dispositivos utilizados para aplicar ambas tecnologías. Como se comentó antes, la \rv utiliza unas gafas que aíslan al usuario de la \realidad. Pero cabe mencionar también que estas gafas son de uso exclusivo para dicha tecnología, así como su elevado coste, ya que son una tecnología que, pese a que ya lleva varios años de desarrollo y mejora, aún necesita asentarse correctamente en el mercado. La \ra, en cambio, se apoya generalmente en dos tecnologías dependiendo de su uso: por un lado, para los usos más cotidianos (aunque a veces también se encuentran en este grupo usos profesionales), se suele implementar en dispositivos móviles, donde la cámara capta las imágenes del entorno y el propio dispositivo móvil añade la información pertinente; por otro lado, y para usos exclusivamente profesionales, muchas empresas han comenzado a utilizar gafas de \ra donde, mediante un juego de espejos, el usuario es capaz de ver información añadida a su entorno.

        \paragraph{}
        Por último, y aunque ya se ha adelantado algo, se podría mencionar también como distinción que, por lo menos hasta el día de hoy, la \rv se está especializando más en el área lúdica, al ser los desarrolladores de videojuegos los principales interesados en esta tecnología, aunque no se descarta que en el futuro pueda aplicarse para motivos más profesionales. La \ra, aunque también se utiliza para este fin, ha conseguido entrar en el área profesional como herramienta para varios sectores, como es el del marketing, donde aplicaciones como la de Ikea permiten al usuario colocar muebles de la tienda en su propia casa para probarlos y, así, impulsar las ventas.
        
\end{document}
