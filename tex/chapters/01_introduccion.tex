\documentclass{subfiles}

\begin{document}

  \chapter{Introducción}
  \label{chap:1}

    \section{Motivación}
    \label{sec:1.1}

      {El \TFG supone el último paso para obtener la titulación cursada, y se muestra de manera muy intimidante al alumno. Por mucho que otros compañeros de años anteriores intenten suavizar su dificultad o la necesidad de hacer algo grande, uno no deja de verlo como la culminación de varios años de aprendizaje en un estudio superior. Por eso, fue difícil elegir algo que realizar.}
      
      \paragraph{}
      {Durante varios años, estuve dispuesto a llevar mi propio proyecto con varias ideas que fui reservando para más adelante, e incluso fueron varias las revisiones que hice a las ideas preparadas para distintos trabajos por distintos profesores. Sin embargo, a medida que se iba acercando la fecha de comenzar este proyecto, iba creciendo la necesidad de comenzar a trabajar y tener mis propios ingresos. Así, y teniendo el \TFG en mente, comencé a buscar empleo planteando a las empresas que me concedían entrevista dos condiciones: media jornada para poder terminar mis estudios y que me ofrecieran un proyecto válido para un \tfg para poder terminar mi carrera.}
      
      \paragraph{}
      {Fue así como entré en \silverstorm, ahora \thirdera después de su adquisición por parte de esta última. Ellos me concedieron la flexibilidad necesaria para compaginar empleo y estudios, así como facilidades a la hora de realizar los exámenes de la Universidad. Además, al entrar en el equipo de innovación, sabía que me podrían ofrecer un proyecto que se distinguiese de los que ofrecían al resto de estudiantes que trabajaban en la empresa.}

      \paragraph{}
      {Tras un tiempo, y después de comentarles que quería comenzar a desarrollar mi \TFG, me ofrecieron el proyecto que se explicará en esta memoria. Por aquel entonces, la página web de \silverstorm desarrollada en \wordpress contaba con un estilo sencillo y familiar para que los usuarios que entrasen conociesen la empresa y pudieran obtener la información que desearan de ella: a qué nos dedicamos, cómo y con qué trabajamos, qué clientes tenemos o hemos tenido, de qué ofertas de trabajo disponemos, etc. No obstante, esta no lograba destacar de otras webs corporativas: sus animaciones eran aquellas que ofrecía \wordpress, los dibujos utilizados eran de una colección de imágenes de stock retocadas y su sencillez implicaba una cierta monotonía.}

\end{document}
